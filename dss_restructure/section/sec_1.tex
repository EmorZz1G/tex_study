\section{\centering 问题重述}
\indent
传统的机器学习方法在求解具有连续动作或状态的问题时其效率较低,使用特殊的微分对策问题可以评估一种机器学习算法的算力。
\zhcrlf 传统的羊-犬博弈问题是一个很好的特殊微分对策问题。羊需要逃出一个半径为$R$的圆形圈,羊的速率为$v$且在逃跑的过程中始终保持不变。羊的逃逸路径上的每一点距离圆心的距离始终不会减少,在此条件下羊具有任意的转弯能力。只要羊逃出圆形圈则获得胜利。犬沿着圆周以定速率$V$围捕羊,任何时刻具有选择圆周两个方向其中之一的能力。
\zhcrlf 请建立数学模型研究一下问题:
\begin{enumerate}[nosep]
    \item 通过运动学精确建模求解犬的最优围堵策略;
    \item 假设犬以最优策略围堵,基于精确建模求解羊可以逃逸胜出的条件;
    \item 假设羊理解自己的能力、限制和躲避犬围堵而逃逸的目标,但不具备基于运动学的最优化决策知识,假设2中羊可以逃逸的条件被满足,给出一种机器学习方法,使得羊通过学习训练后实现逃逸;
    \item 设计一套评价体系,定量评价3中给出的机器学习方法的学习能力;
    \item 提出并定量评价更多的羊逃逸机器学习方法。
\end{enumerate}